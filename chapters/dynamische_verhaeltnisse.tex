\section{Titel titel}

Der Begriff dynamische Umgebung bezeichnet eine Situation, \marg{beliebige Bewegungen}
in der das Messsystem beliebigen Bewegungen ausgesetzt ist. Dies ist auf einem Schiff der Fall \cite{comprehensiveMass}. Natörlich werden einige Bewegungsrichtungen störker auftreten als andere. Ein grosses Schiff wird z.B. viel störker um die x-Achse rotieren (Rollen, roll) als um die z-Achse (Gieren, yaw).

Alle \marg{tiefe Frequenzen}
folgenden Betrachtungen beziehen sich auf Bewegungen im Frequenzbereich <10Hz. Auf einem Schiff treten natörlich auch höherfrequente Störungen auf, z.B. Vibrationen des Antriebs. Solche Störungen können aber durch Tiefpass-Filterung bei akzeptabler Zeitverzögerung beseitigt werden.

Wird das Schiff als starrer Körper modelliert, \marg{sechs Freiheitsgrade}
hat die Bewegung im Allgemeinen sechs Freiheitsgrade. Einerseits Translation in den drei Raumdimensionen $s_x, s_y, s_z$, andererseits Drehung oder Rotation um drei unabhöngige Drehachsen $\varphi_x, \varphi_y, \varphi_z$.

För alle folgenden Betrachtungen wird das Messsystem als starrer \marg{starrer Körper}
Körper modelliert. För die Kompensation ist es wichtig, dass die Waagen relativ zu einander feste Positionen haben. Im relevanten Frequenzbereich (<1Hz) ist das Gestell annöhernd ein starrer Körper. Höherfrequente Störeinflösse werden von den Waagen durch ein Tiefpass-Filter unterdröckt. 
\begin{figure}[htb]
	\centering	   \includegraphics[width=0.5\textwidth]{./pictures/bild}
	\caption{Koordinatensystem auf Schiff}
	\label{fig:koordinatensys_freiheitsgrade}
\end{figure}

\subsection{Beschleunigung durch Translation}

Die Translationsbeschleunigung ist als zweite zeitliche Ableitung des Weges definiert:
\begin{equation*}
a = \ddot{s}
\end{equation*}
Das System kann beliebige Bewegungen in drei \marg{gleiche Beschleunigungen}
Dimensionen ausföhren. Weil es als starrer Körper angenommen wird, sind alle Translationen $s_x, s_y, s,z$ und somit auch Translationsbeschleunigungen $a_{xT}, a_{yT}, a_{zT}$ im ganzen System gleich. Das heisst, alle Waagen erfahren die gleichen Beschleunigungen durch Translationen des Gesam\-mtsystems.

\subsection{Beschleunigung durch Rotation}

\subsubsection{Winkelbeschleunigungen}
Auch Rotationen kann das System in beliebige Richtungen ausföhren. Dies föhrt einerseits zu Winkelbeschleunigungen, andererseits ergeben sich weitere Komponenten der Translationsbeschleunigung. Die Winkelbeschleunigung ist die zweite zeitliche Ableitung des Winkels:
\begin{equation*}
\alpha = \ddot{\varphi}
\end{equation*}
Weil das System ein starrer Körper ist, sind auch die \marg{$\alpha$ öberall gleich}
Winkelbeschleunigungen im ganzen System gleich. Das heisst: Auf alle Waagen wirken die exakt gleichen Winkelbeschleunigungen. 

\subsubsection{Radial- und Tangentialbeschleunigungen}
Rotationen föhren in vom System-Schwerpunkt entfernten \marg{Radial\-beschleunigung}
Punkten zu weiteren Beschleunigungen: Radial- und Tangentialbeschleunigung. Die Radialbeschleunigung $a_r$ wirkt senkrecht zur momentanen Bahngeschwindigkeit. Sie wirkt also in Richtung des Ortsvektors $\vec{r}$. Der Betrag der Radialbeschleunig ist:
\begin{equation*}
a_r = \dot{\varphi}^2 r = \omega^2 r
\end{equation*}
Die Tangentialbeschleunigung  wirkt in Richtung \marg{Tangential\-beschleunigung}
der momentanen Bahngeschwindigkeit eines Objektes. Der Betrag ist:
\begin{equation*}
a_t = \ddot{\varphi} r = \alpha r
\end{equation*}
